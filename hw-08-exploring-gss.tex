% Options for packages loaded elsewhere
\PassOptionsToPackage{unicode}{hyperref}
\PassOptionsToPackage{hyphens}{url}
\documentclass[
]{article}
\usepackage{xcolor}
\usepackage[margin=1in]{geometry}
\usepackage{amsmath,amssymb}
\setcounter{secnumdepth}{5}
\usepackage{iftex}
\ifPDFTeX
  \usepackage[T1]{fontenc}
  \usepackage[utf8]{inputenc}
  \usepackage{textcomp} % provide euro and other symbols
\else % if luatex or xetex
  \usepackage{unicode-math} % this also loads fontspec
  \defaultfontfeatures{Scale=MatchLowercase}
  \defaultfontfeatures[\rmfamily]{Ligatures=TeX,Scale=1}
\fi
\usepackage{lmodern}
\ifPDFTeX\else
  % xetex/luatex font selection
\fi
% Use upquote if available, for straight quotes in verbatim environments
\IfFileExists{upquote.sty}{\usepackage{upquote}}{}
\IfFileExists{microtype.sty}{% use microtype if available
  \usepackage[]{microtype}
  \UseMicrotypeSet[protrusion]{basicmath} % disable protrusion for tt fonts
}{}
\makeatletter
\@ifundefined{KOMAClassName}{% if non-KOMA class
  \IfFileExists{parskip.sty}{%
    \usepackage{parskip}
  }{% else
    \setlength{\parindent}{0pt}
    \setlength{\parskip}{6pt plus 2pt minus 1pt}}
}{% if KOMA class
  \KOMAoptions{parskip=half}}
\makeatother
\usepackage{color}
\usepackage{fancyvrb}
\newcommand{\VerbBar}{|}
\newcommand{\VERB}{\Verb[commandchars=\\\{\}]}
\DefineVerbatimEnvironment{Highlighting}{Verbatim}{commandchars=\\\{\}}
% Add ',fontsize=\small' for more characters per line
\usepackage{framed}
\definecolor{shadecolor}{RGB}{248,248,248}
\newenvironment{Shaded}{\begin{snugshade}}{\end{snugshade}}
\newcommand{\AlertTok}[1]{\textcolor[rgb]{0.94,0.16,0.16}{#1}}
\newcommand{\AnnotationTok}[1]{\textcolor[rgb]{0.56,0.35,0.01}{\textbf{\textit{#1}}}}
\newcommand{\AttributeTok}[1]{\textcolor[rgb]{0.13,0.29,0.53}{#1}}
\newcommand{\BaseNTok}[1]{\textcolor[rgb]{0.00,0.00,0.81}{#1}}
\newcommand{\BuiltInTok}[1]{#1}
\newcommand{\CharTok}[1]{\textcolor[rgb]{0.31,0.60,0.02}{#1}}
\newcommand{\CommentTok}[1]{\textcolor[rgb]{0.56,0.35,0.01}{\textit{#1}}}
\newcommand{\CommentVarTok}[1]{\textcolor[rgb]{0.56,0.35,0.01}{\textbf{\textit{#1}}}}
\newcommand{\ConstantTok}[1]{\textcolor[rgb]{0.56,0.35,0.01}{#1}}
\newcommand{\ControlFlowTok}[1]{\textcolor[rgb]{0.13,0.29,0.53}{\textbf{#1}}}
\newcommand{\DataTypeTok}[1]{\textcolor[rgb]{0.13,0.29,0.53}{#1}}
\newcommand{\DecValTok}[1]{\textcolor[rgb]{0.00,0.00,0.81}{#1}}
\newcommand{\DocumentationTok}[1]{\textcolor[rgb]{0.56,0.35,0.01}{\textbf{\textit{#1}}}}
\newcommand{\ErrorTok}[1]{\textcolor[rgb]{0.64,0.00,0.00}{\textbf{#1}}}
\newcommand{\ExtensionTok}[1]{#1}
\newcommand{\FloatTok}[1]{\textcolor[rgb]{0.00,0.00,0.81}{#1}}
\newcommand{\FunctionTok}[1]{\textcolor[rgb]{0.13,0.29,0.53}{\textbf{#1}}}
\newcommand{\ImportTok}[1]{#1}
\newcommand{\InformationTok}[1]{\textcolor[rgb]{0.56,0.35,0.01}{\textbf{\textit{#1}}}}
\newcommand{\KeywordTok}[1]{\textcolor[rgb]{0.13,0.29,0.53}{\textbf{#1}}}
\newcommand{\NormalTok}[1]{#1}
\newcommand{\OperatorTok}[1]{\textcolor[rgb]{0.81,0.36,0.00}{\textbf{#1}}}
\newcommand{\OtherTok}[1]{\textcolor[rgb]{0.56,0.35,0.01}{#1}}
\newcommand{\PreprocessorTok}[1]{\textcolor[rgb]{0.56,0.35,0.01}{\textit{#1}}}
\newcommand{\RegionMarkerTok}[1]{#1}
\newcommand{\SpecialCharTok}[1]{\textcolor[rgb]{0.81,0.36,0.00}{\textbf{#1}}}
\newcommand{\SpecialStringTok}[1]{\textcolor[rgb]{0.31,0.60,0.02}{#1}}
\newcommand{\StringTok}[1]{\textcolor[rgb]{0.31,0.60,0.02}{#1}}
\newcommand{\VariableTok}[1]{\textcolor[rgb]{0.00,0.00,0.00}{#1}}
\newcommand{\VerbatimStringTok}[1]{\textcolor[rgb]{0.31,0.60,0.02}{#1}}
\newcommand{\WarningTok}[1]{\textcolor[rgb]{0.56,0.35,0.01}{\textbf{\textit{#1}}}}
\usepackage{graphicx}
\makeatletter
\newsavebox\pandoc@box
\newcommand*\pandocbounded[1]{% scales image to fit in text height/width
  \sbox\pandoc@box{#1}%
  \Gscale@div\@tempa{\textheight}{\dimexpr\ht\pandoc@box+\dp\pandoc@box\relax}%
  \Gscale@div\@tempb{\linewidth}{\wd\pandoc@box}%
  \ifdim\@tempb\p@<\@tempa\p@\let\@tempa\@tempb\fi% select the smaller of both
  \ifdim\@tempa\p@<\p@\scalebox{\@tempa}{\usebox\pandoc@box}%
  \else\usebox{\pandoc@box}%
  \fi%
}
% Set default figure placement to htbp
\def\fps@figure{htbp}
\makeatother
\setlength{\emergencystretch}{3em} % prevent overfull lines
\providecommand{\tightlist}{%
  \setlength{\itemsep}{0pt}\setlength{\parskip}{0pt}}
\usepackage[]{natbib}
\bibliographystyle{plainnat}
\usepackage{bookmark}
\IfFileExists{xurl.sty}{\usepackage{xurl}}{} % add URL line breaks if available
\urlstyle{same}
\hypersetup{
  pdftitle={HW 08 - Exploring the GSS},
  hidelinks,
  pdfcreator={LaTeX via pandoc}}

\title{HW 08 - Exploring the GSS}
\author{}
\date{\vspace{-2.5em}}

\begin{document}
\maketitle

{
\setcounter{tocdepth}{2}
\tableofcontents
}
\begin{figure}
\includegraphics[width=0.8\linewidth]{img/mauro-mora-31-pOduwZGE-unsplash} \caption{Photo by Mauro Mora on Unsplash}\label{fig:photo}
\end{figure}

The GSS gathers data on contemporary American society in order to
monitor and explain trends and constants in attitudes, behaviours, and
attributes. Hundreds of trends have been tracked since 1972. In
addition, since the GSS adopted questions from earlier surveys, trends
can be followed for up to 70 years.

The GSS contains a standard core of demographic, behavioral, and
attitudinal questions, plus topics of special interest. Among the topics
covered are civil liberties, crime and violence, intergroup tolerance,
morality, national spending priorities, psychological well-being, social
mobility, and stress and traumatic events.

In this assignment we analyze data from the 2016 GSS, using it to
estimate values of population parameters of interest about US
adults.\footnote{Smith, Tom W, Peter Marsden, Michael Hout, and Jibum
  Kim. General Social Surveys, 1972-2016 {[}machine-readable data
  file{]} /Principal Investigator, Tom W. Smith; Co-Principal
  Investigator, Peter V. Marsden; Co-Principal Investigator, Michael
  Hout; Sponsored by National Science Foundation. -NORC ed.- Chicago:
  NORC at the University of Chicago {[}producer and distributor{]}. Data
  accessed from the GSS Data Explorer website at
  gssdataexplorer.norc.org.}

\section{Getting started}\label{getting-started}

Go to the course GitHub organization and locate your homework repo,
clone it in Posit Cloud and open the R Markdown document. Knit the
document to make sure it compiles without errors.

\subsection{Warm up}\label{warm-up}

Before we introduce the data, let's warm up with some simple exercises.
Update the YAML of your R Markdown file with your information, knit,
commit, and push your changes. Make sure to commit with a meaningful
commit message. Then, go to your repo on GitHub and confirm that your
changes are visible in your Rmd \textbf{and} md files. If anything is
missing, commit and push again.

\subsection{Packages}\label{packages}

We'll use the \textbf{tidyverse} package for much of the data wrangling
and visualisation and the data lives in the \textbf{dsbox} package.
These packages are already installed for you. You can load them by
running the following in your Console:

\begin{Shaded}
\begin{Highlighting}[]
\FunctionTok{library}\NormalTok{(tidyverse)}
\FunctionTok{library}\NormalTok{(dsbox)}
\FunctionTok{library}\NormalTok{(broom)}
\end{Highlighting}
\end{Shaded}

\subsection{Data}\label{data}

The data can be found in the \textbf{dsbox} package, and it's called
\texttt{gss16}. Since the dataset is distributed with the package, we
don't need to load it separately; it becomes available to us when we
load the package. You can find out more about the dataset by inspecting
its documentation, which you can access by running \texttt{?gss16} in
the Console or using the Help menu in RStudio to search for
\texttt{gss16}. You can also find this information
\href{https://rstudio-education.github.io/dsbox/reference/gss16.html}{here}.

\section{Exercises}\label{exercises}

\subsection{Part 1: Harassment at work}\label{part-1-harassment-at-work}

In 2016, the GSS added a new question on harassment at work. The
question is phrased as the following.

\begin{quote}
\emph{Over the past five years, have you been harassed by your superiors
or co-workers at your job, for example, have you experienced any
bullying, physical or psychological abuse?}
\end{quote}

Answers to this question are stored in the \texttt{harass5} variable in
our dataset.

\begin{enumerate}
\def\labelenumi{\arabic{enumi}.}
\tightlist
\item
  What are the possible responses to this question and how many
  respondents chose each of these answers?
\end{enumerate}

\begin{Shaded}
\begin{Highlighting}[]
\CommentTok{\# Examine the possible responses to the harassment question}
\NormalTok{gss16 }\SpecialCharTok{\%\textgreater{}\%}
  \FunctionTok{count}\NormalTok{(harass5) }\SpecialCharTok{\%\textgreater{}\%}
  \FunctionTok{arrange}\NormalTok{(}\FunctionTok{desc}\NormalTok{(n))}
\end{Highlighting}
\end{Shaded}

\begin{verbatim}
## # A tibble: 4 x 2
##   harass5                                                     n
##   <chr>                                                   <int>
## 1 <NA>                                                     1398
## 2 No                                                       1136
## 3 Yes                                                       237
## 4 Does not apply (i do not have a job/superior/co-worker)    96
\end{verbatim}

\textbf{Answer:} The possible responses to the harassment question are:
``Yes'' (has been harassed), ``No'' (has not been harassed), ``Does not
apply'' (respondent does not work), and ``NA'' (missing responses). The
distribution shows how many respondents chose each response, with the
majority likely answering ``No'' if workplace harassment is not
widespread.

\begin{enumerate}
\def\labelenumi{\arabic{enumi}.}
\setcounter{enumi}{1}
\tightlist
\item
  What percent of the respondents for whom this question is applicable\\
  (i.e.~excluding \texttt{NA}s and \texttt{Does\ not\ apply}s) have been
  harassed by their superiors or co-workers at their job.
\end{enumerate}

\begin{Shaded}
\begin{Highlighting}[]
\CommentTok{\# Filter to applicable responses only (exclude NA and "Does not apply")}
\NormalTok{harassment\_applicable }\OtherTok{\textless{}{-}}\NormalTok{ gss16 }\SpecialCharTok{\%\textgreater{}\%}
  \FunctionTok{filter}\NormalTok{(}\SpecialCharTok{!}\FunctionTok{is.na}\NormalTok{(harass5), harass5 }\SpecialCharTok{!=} \StringTok{"Does not apply"}\NormalTok{)}

\CommentTok{\# Calculate the percentage of "Yes" responses}
\NormalTok{harassment\_pct }\OtherTok{\textless{}{-}}\NormalTok{ harassment\_applicable }\SpecialCharTok{\%\textgreater{}\%}
  \FunctionTok{summarise}\NormalTok{(}
    \AttributeTok{n\_total =} \FunctionTok{n}\NormalTok{(),}
    \AttributeTok{n\_yes =} \FunctionTok{sum}\NormalTok{(harass5 }\SpecialCharTok{==} \StringTok{"Yes"}\NormalTok{),}
    \AttributeTok{pct\_harassed =}\NormalTok{ (n\_yes }\SpecialCharTok{/}\NormalTok{ n\_total) }\SpecialCharTok{*} \DecValTok{100}
\NormalTok{  )}

\NormalTok{harassment\_pct }\SpecialCharTok{\%\textgreater{}\%}
  \FunctionTok{print}\NormalTok{()}
\end{Highlighting}
\end{Shaded}

\begin{verbatim}
## # A tibble: 1 x 3
##   n_total n_yes pct_harassed
##     <int> <int>        <dbl>
## 1    1469   237         16.1
\end{verbatim}

\begin{Shaded}
\begin{Highlighting}[]
\FunctionTok{cat}\NormalTok{(}
  \StringTok{"}\SpecialCharTok{\textbackslash{}n}\StringTok{Answer: Among respondents for whom this question is applicable,"}\NormalTok{,}
  \FunctionTok{round}\NormalTok{(harassment\_pct}\SpecialCharTok{$}\NormalTok{pct\_harassed, }\DecValTok{1}\NormalTok{), }\StringTok{"\% have been harassed at work.}\SpecialCharTok{\textbackslash{}n}\StringTok{"}
\NormalTok{)}
\end{Highlighting}
\end{Shaded}

\begin{verbatim}
## 
## Answer: Among respondents for whom this question is applicable, 16.1 % have been harassed at work.
\end{verbatim}

\textbf{Interpretation:} This percentage represents the proportion of
employed respondents who have experienced workplace harassment in the
past five years, providing a measure of the prevalence of bullying and
abuse in the US workplace according to the 2016 GSS.

Knit, \emph{commit, and push your changes to GitHub with an appropriate
commit message. Make sure to commit and push all changed files so that
your Git pane is cleared up afterwards.}

\subsection{Part 2: Time spent on
email}\label{part-2-time-spent-on-email}

The 2016 GSS also asked respondents how many hours and minutes they
spend on email weekly. The responses to these questions are recorded in
the \texttt{emailhr} and \texttt{emailmin} variables. For example, if
the response is 2.5 hrs, this would be recorded as
\texttt{emailhr\ =\ 2} and \texttt{emailmin\ =\ 30}.

\begin{enumerate}
\def\labelenumi{\arabic{enumi}.}
\setcounter{enumi}{2}
\tightlist
\item
  Create a new variable called \texttt{email} that combines these two
  variables to reports the number of minutes the respondents spend on
  email weekly.
\end{enumerate}

\begin{Shaded}
\begin{Highlighting}[]
\CommentTok{\# Create email variable combining hours and minutes}
\NormalTok{gss16 }\OtherTok{\textless{}{-}}\NormalTok{ gss16 }\SpecialCharTok{\%\textgreater{}\%}
  \FunctionTok{mutate}\NormalTok{(}\AttributeTok{email =}\NormalTok{ emailhr }\SpecialCharTok{*} \DecValTok{60} \SpecialCharTok{+}\NormalTok{ emailmin)}

\CommentTok{\# Check the new variable}
\NormalTok{gss16 }\SpecialCharTok{\%\textgreater{}\%}
  \FunctionTok{select}\NormalTok{(emailhr, emailmin, email) }\SpecialCharTok{\%\textgreater{}\%}
  \FunctionTok{head}\NormalTok{(}\DecValTok{10}\NormalTok{)}
\end{Highlighting}
\end{Shaded}

\begin{verbatim}
## # A tibble: 10 x 3
##    emailhr emailmin email
##      <dbl>    <dbl> <dbl>
##  1      12        0   720
##  2       0       30    30
##  3      NA       NA    NA
##  4       0       10    10
##  5      NA       NA    NA
##  6       2        0   120
##  7      40        0  2400
##  8      NA       NA    NA
##  9       0        0     0
## 10      NA       NA    NA
\end{verbatim}

\begin{Shaded}
\begin{Highlighting}[]
\FunctionTok{cat}\NormalTok{(}\StringTok{"Email time statistics:}\SpecialCharTok{\textbackslash{}n}\StringTok{"}\NormalTok{)}
\end{Highlighting}
\end{Shaded}

\begin{verbatim}
## Email time statistics:
\end{verbatim}

\begin{Shaded}
\begin{Highlighting}[]
\NormalTok{gss16 }\SpecialCharTok{\%\textgreater{}\%}
  \FunctionTok{summarise}\NormalTok{(}
    \AttributeTok{n\_valid =} \FunctionTok{sum}\NormalTok{(}\SpecialCharTok{!}\FunctionTok{is.na}\NormalTok{(email)),}
    \AttributeTok{mean\_email =} \FunctionTok{mean}\NormalTok{(email, }\AttributeTok{na.rm =} \ConstantTok{TRUE}\NormalTok{),}
    \AttributeTok{median\_email =} \FunctionTok{median}\NormalTok{(email, }\AttributeTok{na.rm =} \ConstantTok{TRUE}\NormalTok{),}
    \AttributeTok{min\_email =} \FunctionTok{min}\NormalTok{(email, }\AttributeTok{na.rm =} \ConstantTok{TRUE}\NormalTok{),}
    \AttributeTok{max\_email =} \FunctionTok{max}\NormalTok{(email, }\AttributeTok{na.rm =} \ConstantTok{TRUE}\NormalTok{)}
\NormalTok{  ) }\SpecialCharTok{\%\textgreater{}\%}
  \FunctionTok{print}\NormalTok{()}
\end{Highlighting}
\end{Shaded}

\begin{verbatim}
## # A tibble: 1 x 5
##   n_valid mean_email median_email min_email max_email
##     <int>      <dbl>        <dbl>     <dbl>     <dbl>
## 1    1649       417.          120         0      6000
\end{verbatim}

\textbf{Explanation:} The \texttt{email} variable is created by
converting hours to minutes (multiplying by 60) and adding the minutes,
resulting in total minutes spent on email per week.

\begin{enumerate}
\def\labelenumi{\arabic{enumi}.}
\setcounter{enumi}{3}
\tightlist
\item
  Visualize the distribution of this new variable. Find the mean and the
  median number of minutes respondents spend on email weekly. Is the
  mean or the median a better measure of the typical among of time
  Americans spend on email weekly? Why?
\end{enumerate}

\begin{Shaded}
\begin{Highlighting}[]
\CommentTok{\# Visualize the distribution}
\NormalTok{gss16 }\SpecialCharTok{\%\textgreater{}\%}
  \FunctionTok{filter}\NormalTok{(}\SpecialCharTok{!}\FunctionTok{is.na}\NormalTok{(email)) }\SpecialCharTok{\%\textgreater{}\%}
  \FunctionTok{ggplot}\NormalTok{(}\FunctionTok{aes}\NormalTok{(}\AttributeTok{x =}\NormalTok{ email)) }\SpecialCharTok{+}
  \FunctionTok{geom\_histogram}\NormalTok{(}\AttributeTok{binwidth =} \DecValTok{60}\NormalTok{, }\AttributeTok{fill =} \StringTok{"steelblue"}\NormalTok{, }\AttributeTok{alpha =} \FloatTok{0.7}\NormalTok{) }\SpecialCharTok{+}
  \FunctionTok{labs}\NormalTok{(}
    \AttributeTok{title =} \StringTok{"Distribution of Weekly Email Time"}\NormalTok{,}
    \AttributeTok{x =} \StringTok{"Minutes per Week"}\NormalTok{,}
    \AttributeTok{y =} \StringTok{"Count"}
\NormalTok{  ) }\SpecialCharTok{+}
  \FunctionTok{theme\_minimal}\NormalTok{()}
\end{Highlighting}
\end{Shaded}

\includegraphics[width=0.8\linewidth]{hw-08-exploring-gss_files/figure-latex/exercise-4-1}

\begin{Shaded}
\begin{Highlighting}[]
\CommentTok{\# Calculate summary statistics}
\NormalTok{email\_stats }\OtherTok{\textless{}{-}}\NormalTok{ gss16 }\SpecialCharTok{\%\textgreater{}\%}
  \FunctionTok{filter}\NormalTok{(}\SpecialCharTok{!}\FunctionTok{is.na}\NormalTok{(email)) }\SpecialCharTok{\%\textgreater{}\%}
  \FunctionTok{summarise}\NormalTok{(}
    \AttributeTok{mean\_email =} \FunctionTok{mean}\NormalTok{(email),}
    \AttributeTok{median\_email =} \FunctionTok{median}\NormalTok{(email),}
    \AttributeTok{sd\_email =} \FunctionTok{sd}\NormalTok{(email),}
    \AttributeTok{q1 =} \FunctionTok{quantile}\NormalTok{(email, }\FloatTok{0.25}\NormalTok{),}
    \AttributeTok{q3 =} \FunctionTok{quantile}\NormalTok{(email, }\FloatTok{0.75}\NormalTok{)}
\NormalTok{  )}

\FunctionTok{cat}\NormalTok{(}\StringTok{"Email time summary:}\SpecialCharTok{\textbackslash{}n}\StringTok{"}\NormalTok{)}
\end{Highlighting}
\end{Shaded}

\begin{verbatim}
## Email time summary:
\end{verbatim}

\begin{Shaded}
\begin{Highlighting}[]
\FunctionTok{print}\NormalTok{(email\_stats)}
\end{Highlighting}
\end{Shaded}

\begin{verbatim}
## # A tibble: 1 x 5
##   mean_email median_email sd_email    q1    q3
##        <dbl>        <dbl>    <dbl> <dbl> <dbl>
## 1       417.          120     680.    50   480
\end{verbatim}

\begin{Shaded}
\begin{Highlighting}[]
\CommentTok{\# Store values for interpretation}
\NormalTok{mean\_val }\OtherTok{\textless{}{-}} \FunctionTok{round}\NormalTok{(email\_stats}\SpecialCharTok{$}\NormalTok{mean\_email, }\DecValTok{1}\NormalTok{)}
\NormalTok{median\_val }\OtherTok{\textless{}{-}} \FunctionTok{round}\NormalTok{(email\_stats}\SpecialCharTok{$}\NormalTok{median\_email, }\DecValTok{1}\NormalTok{)}
\end{Highlighting}
\end{Shaded}

\textbf{Answer:} The mean is approximately 416.8 minutes per week, while
the median is approximately 120 minutes per week.

\textbf{Which measure is better and why?} The median is likely a better
measure of the typical amount of time Americans spend on email weekly.
Email time data is typically right-skewed (some people spend many hours
on email, pulling the mean upward), so the median better represents the
typical respondent's experience. The median is more robust to extreme
values (outliers) and provides a more representative picture of the
``typical'' employee.

\begin{enumerate}
\def\labelenumi{\arabic{enumi}.}
\setcounter{enumi}{4}
\tightlist
\item
  Create another new variable, \texttt{snap\_insta} that is coded as
  ``Yes'' if the respondent reported using any of Snapchat
  (\texttt{snapchat}) or Instagram (\texttt{instagrm}), and ``No'' if
  not. If the recorded value was \texttt{NA} for both of these
  questions, the value in your new variable should also be \texttt{NA}.
\end{enumerate}

\begin{Shaded}
\begin{Highlighting}[]
\CommentTok{\# Create snap\_insta variable}
\NormalTok{gss16 }\OtherTok{\textless{}{-}}\NormalTok{ gss16 }\SpecialCharTok{\%\textgreater{}\%}
  \FunctionTok{mutate}\NormalTok{(}\AttributeTok{snap\_insta =} \FunctionTok{case\_when}\NormalTok{(}
    \FunctionTok{is.na}\NormalTok{(snapchat) }\SpecialCharTok{\&} \FunctionTok{is.na}\NormalTok{(instagrm) }\SpecialCharTok{\textasciitilde{}} \ConstantTok{NA\_character\_}\NormalTok{,}
\NormalTok{    snapchat }\SpecialCharTok{==} \StringTok{"Yes"} \SpecialCharTok{|}\NormalTok{ instagrm }\SpecialCharTok{==} \StringTok{"Yes"} \SpecialCharTok{\textasciitilde{}} \StringTok{"Yes"}\NormalTok{,}
    \ConstantTok{TRUE} \SpecialCharTok{\textasciitilde{}} \StringTok{"No"}
\NormalTok{  ))}

\CommentTok{\# Verify the new variable}
\NormalTok{gss16 }\SpecialCharTok{\%\textgreater{}\%}
  \FunctionTok{select}\NormalTok{(snapchat, instagrm, snap\_insta) }\SpecialCharTok{\%\textgreater{}\%}
  \FunctionTok{filter}\NormalTok{(}\SpecialCharTok{!}\NormalTok{(}\FunctionTok{is.na}\NormalTok{(snapchat) }\SpecialCharTok{\&} \FunctionTok{is.na}\NormalTok{(instagrm))) }\SpecialCharTok{\%\textgreater{}\%}
  \FunctionTok{head}\NormalTok{(}\DecValTok{15}\NormalTok{)}
\end{Highlighting}
\end{Shaded}

\begin{verbatim}
## # A tibble: 15 x 3
##    snapchat instagrm snap_insta
##    <chr>    <chr>    <chr>     
##  1 No       No       No        
##  2 No       No       No        
##  3 Yes      Yes      Yes       
##  4 No       Yes      Yes       
##  5 Yes      Yes      Yes       
##  6 No       No       No        
##  7 No       No       No        
##  8 No       No       No        
##  9 No       No       No        
## 10 No       No       No        
## 11 Yes      No       Yes       
## 12 No       No       No        
## 13 No       Yes      Yes       
## 14 Yes      Yes      Yes       
## 15 No       Yes      Yes
\end{verbatim}

\begin{Shaded}
\begin{Highlighting}[]
\FunctionTok{cat}\NormalTok{(}\StringTok{"snap\_insta distribution:}\SpecialCharTok{\textbackslash{}n}\StringTok{"}\NormalTok{)}
\end{Highlighting}
\end{Shaded}

\begin{verbatim}
## snap_insta distribution:
\end{verbatim}

\begin{Shaded}
\begin{Highlighting}[]
\NormalTok{gss16 }\SpecialCharTok{\%\textgreater{}\%}
  \FunctionTok{count}\NormalTok{(snap\_insta)}
\end{Highlighting}
\end{Shaded}

\begin{verbatim}
## # A tibble: 3 x 2
##   snap_insta     n
##   <chr>      <int>
## 1 No           858
## 2 Yes          514
## 3 <NA>        1495
\end{verbatim}

\textbf{Explanation:} The \texttt{snap\_insta} variable combines
information from both Snapchat and Instagram use. If either is ``Yes'',
the new variable is ``Yes''. If both are ``No'', it's ``No''. If both
are missing, it's also missing.

\begin{enumerate}
\def\labelenumi{\arabic{enumi}.}
\setcounter{enumi}{5}
\tightlist
\item
  Calculate the percentage of Yes's for \texttt{snap\_insta} among those
  who answered the question, i.e.~excluding \texttt{NA}s.
\end{enumerate}

\begin{Shaded}
\begin{Highlighting}[]
\CommentTok{\# Calculate percentage of Yes responses}
\NormalTok{snap\_insta\_pct }\OtherTok{\textless{}{-}}\NormalTok{ gss16 }\SpecialCharTok{\%\textgreater{}\%}
  \FunctionTok{filter}\NormalTok{(}\SpecialCharTok{!}\FunctionTok{is.na}\NormalTok{(snap\_insta)) }\SpecialCharTok{\%\textgreater{}\%}
  \FunctionTok{summarise}\NormalTok{(}
    \AttributeTok{n\_total =} \FunctionTok{n}\NormalTok{(),}
    \AttributeTok{n\_yes =} \FunctionTok{sum}\NormalTok{(snap\_insta }\SpecialCharTok{==} \StringTok{"Yes"}\NormalTok{),}
    \AttributeTok{pct\_yes =}\NormalTok{ (n\_yes }\SpecialCharTok{/}\NormalTok{ n\_total) }\SpecialCharTok{*} \DecValTok{100}
\NormalTok{  )}

\FunctionTok{cat}\NormalTok{(}\StringTok{"Snapchat/Instagram usage among respondents:}\SpecialCharTok{\textbackslash{}n}\StringTok{"}\NormalTok{)}
\end{Highlighting}
\end{Shaded}

\begin{verbatim}
## Snapchat/Instagram usage among respondents:
\end{verbatim}

\begin{Shaded}
\begin{Highlighting}[]
\FunctionTok{print}\NormalTok{(snap\_insta\_pct)}
\end{Highlighting}
\end{Shaded}

\begin{verbatim}
## # A tibble: 1 x 3
##   n_total n_yes pct_yes
##     <int> <int>   <dbl>
## 1    1372   514    37.5
\end{verbatim}

\begin{Shaded}
\begin{Highlighting}[]
\FunctionTok{cat}\NormalTok{(}
  \StringTok{"}\SpecialCharTok{\textbackslash{}n}\StringTok{Answer: Among respondents who answered the question,"}\NormalTok{,}
  \FunctionTok{round}\NormalTok{(snap\_insta\_pct}\SpecialCharTok{$}\NormalTok{pct\_yes, }\DecValTok{1}\NormalTok{), }\StringTok{"\% use Snapchat or Instagram.}\SpecialCharTok{\textbackslash{}n}\StringTok{"}
\NormalTok{)}
\end{Highlighting}
\end{Shaded}

\begin{verbatim}
## 
## Answer: Among respondents who answered the question, 37.5 % use Snapchat or Instagram.
\end{verbatim}

\textbf{Interpretation:} This percentage indicates the adoption rate of
these social media platforms among GSS respondents in 2016.

\begin{enumerate}
\def\labelenumi{\arabic{enumi}.}
\setcounter{enumi}{6}
\tightlist
\item
  What are the possible responses to the question \emph{Last week were
  you working full time, part time, going to school, keeping house, or
  what?} and how many respondents chose each of these answers? Note that
  this information is stored in the \texttt{wrkstat} variable.
\end{enumerate}

\begin{Shaded}
\begin{Highlighting}[]
\CommentTok{\# Examine work status variable}
\NormalTok{gss16 }\SpecialCharTok{\%\textgreater{}\%}
  \FunctionTok{count}\NormalTok{(wrkstat) }\SpecialCharTok{\%\textgreater{}\%}
  \FunctionTok{arrange}\NormalTok{(}\FunctionTok{desc}\NormalTok{(n))}
\end{Highlighting}
\end{Shaded}

\begin{verbatim}
## # A tibble: 9 x 2
##   wrkstat              n
##   <chr>            <int>
## 1 Working fulltime  1321
## 2 Retired            574
## 3 Working parttime   345
## 4 Keeping house      284
## 5 Unempl, laid off   118
## 6 Other               89
## 7 School              76
## 8 Temp not working    57
## 9 <NA>                 3
\end{verbatim}

\begin{Shaded}
\begin{Highlighting}[]
\FunctionTok{cat}\NormalTok{(}\StringTok{"Work status categories and frequencies:}\SpecialCharTok{\textbackslash{}n}\StringTok{"}\NormalTok{)}
\end{Highlighting}
\end{Shaded}

\begin{verbatim}
## Work status categories and frequencies:
\end{verbatim}

\begin{Shaded}
\begin{Highlighting}[]
\NormalTok{gss16 }\SpecialCharTok{\%\textgreater{}\%}
  \FunctionTok{count}\NormalTok{(wrkstat) }\SpecialCharTok{\%\textgreater{}\%}
  \FunctionTok{print}\NormalTok{()}
\end{Highlighting}
\end{Shaded}

\begin{verbatim}
## # A tibble: 9 x 2
##   wrkstat              n
##   <chr>            <int>
## 1 Keeping house      284
## 2 Other               89
## 3 Retired            574
## 4 School              76
## 5 Temp not working    57
## 6 Unempl, laid off   118
## 7 Working fulltime  1321
## 8 Working parttime   345
## 9 <NA>                 3
\end{verbatim}

\textbf{Answer:} The possible responses to the work status question
include categories such as ``Full-time'', ``Part-time'', ``Temporarily
not working'', ``Unemployed - laid off'', ``Retired'', ``In school'',
``Keeping house'', and possibly others. The counts show how many
respondents fall into each category.

\begin{enumerate}
\def\labelenumi{\arabic{enumi}.}
\setcounter{enumi}{7}
\tightlist
\item
  Fit a model predicting \texttt{email} (number of minutes per week
  spent on email) from \texttt{educ} (number of years of education),
  \texttt{wrkstat}, and \texttt{snap\_insta}. Interpret the slopes for
  each of these variables.
\end{enumerate}

\begin{Shaded}
\begin{Highlighting}[]
\CommentTok{\# Fit linear regression model}
\NormalTok{email\_model }\OtherTok{\textless{}{-}} \FunctionTok{lm}\NormalTok{(email }\SpecialCharTok{\textasciitilde{}}\NormalTok{ educ }\SpecialCharTok{+}\NormalTok{ wrkstat }\SpecialCharTok{+}\NormalTok{ snap\_insta, }\AttributeTok{data =}\NormalTok{ gss16)}

\CommentTok{\# View model summary}
\FunctionTok{summary}\NormalTok{(email\_model)}
\end{Highlighting}
\end{Shaded}

\begin{verbatim}
## 
## Call:
## lm(formula = email ~ educ + wrkstat + snap_insta, data = gss16)
## 
## Residuals:
##    Min     1Q Median     3Q    Max 
## -760.5 -372.7 -161.2   95.4 3355.6 
## 
## Coefficients:
##                         Estimate Std. Error t value Pr(>|t|)    
## (Intercept)             -229.736    149.837  -1.533  0.12569    
## educ                      29.632      9.601   3.087  0.00211 ** 
## wrkstatOther              33.057    209.470   0.158  0.87465    
## wrkstatRetired            68.279    111.051   0.615  0.53887    
## wrkstatSchool           -123.812    143.981  -0.860  0.39014    
## wrkstatTemp not working  -73.709    153.948  -0.479  0.63225    
## wrkstatUnempl, laid off  118.349    151.242   0.783  0.43419    
## wrkstatWorking fulltime  366.840     87.690   4.183 3.26e-05 ***
## wrkstatWorking parttime   18.900    101.632   0.186  0.85253    
## snap_instaYes            149.961     52.745   2.843  0.00460 ** 
## ---
## Signif. codes:  0 '***' 0.001 '**' 0.01 '*' 0.05 '.' 0.1 ' ' 1
## 
## Residual standard error: 642.2 on 669 degrees of freedom
##   (2188 observations deleted due to missingness)
## Multiple R-squared:  0.1043, Adjusted R-squared:  0.09227 
## F-statistic: 8.657 on 9 and 669 DF,  p-value: 2.395e-12
\end{verbatim}

\begin{Shaded}
\begin{Highlighting}[]
\CommentTok{\# Extract and interpret coefficients}
\NormalTok{coef\_summary }\OtherTok{\textless{}{-}}\NormalTok{ broom}\SpecialCharTok{::}\FunctionTok{tidy}\NormalTok{(email\_model)}
\FunctionTok{print}\NormalTok{(coef\_summary)}
\end{Highlighting}
\end{Shaded}

\begin{verbatim}
## # A tibble: 10 x 5
##    term                    estimate std.error statistic   p.value
##    <chr>                      <dbl>     <dbl>     <dbl>     <dbl>
##  1 (Intercept)               -230.     150.      -1.53  0.126    
##  2 educ                        29.6      9.60     3.09  0.00211  
##  3 wrkstatOther                33.1    209.       0.158 0.875    
##  4 wrkstatRetired              68.3    111.       0.615 0.539    
##  5 wrkstatSchool             -124.     144.      -0.860 0.390    
##  6 wrkstatTemp not working    -73.7    154.      -0.479 0.632    
##  7 wrkstatUnempl, laid off    118.     151.       0.783 0.434    
##  8 wrkstatWorking fulltime    367.      87.7      4.18  0.0000326
##  9 wrkstatWorking parttime     18.9    102.       0.186 0.853    
## 10 snap_instaYes              150.      52.7      2.84  0.00460
\end{verbatim}

\begin{Shaded}
\begin{Highlighting}[]
\FunctionTok{cat}\NormalTok{(}\StringTok{"}\SpecialCharTok{\textbackslash{}n}\StringTok{Model Interpretation:}\SpecialCharTok{\textbackslash{}n}\StringTok{"}\NormalTok{)}
\end{Highlighting}
\end{Shaded}

\begin{verbatim}
## 
## Model Interpretation:
\end{verbatim}

\begin{Shaded}
\begin{Highlighting}[]
\FunctionTok{cat}\NormalTok{(}\StringTok{"Intercept: The baseline email time when other variables are at reference levels.}\SpecialCharTok{\textbackslash{}n}\StringTok{"}\NormalTok{)}
\end{Highlighting}
\end{Shaded}

\begin{verbatim}
## Intercept: The baseline email time when other variables are at reference levels.
\end{verbatim}

\begin{Shaded}
\begin{Highlighting}[]
\FunctionTok{cat}\NormalTok{(}
  \StringTok{"Educ coefficient: For each additional year of education, email time changes by"}\NormalTok{,}
  \FunctionTok{round}\NormalTok{(coef\_summary}\SpecialCharTok{$}\NormalTok{estimate[}\DecValTok{2}\NormalTok{], }\DecValTok{2}\NormalTok{), }\StringTok{"minutes per week.}\SpecialCharTok{\textbackslash{}n}\StringTok{"}
\NormalTok{)}
\end{Highlighting}
\end{Shaded}

\begin{verbatim}
## Educ coefficient: For each additional year of education, email time changes by 29.63 minutes per week.
\end{verbatim}

\begin{Shaded}
\begin{Highlighting}[]
\FunctionTok{cat}\NormalTok{(}\StringTok{"Wrkstat coefficients: Each work status category differs from the reference category.}\SpecialCharTok{\textbackslash{}n}\StringTok{"}\NormalTok{)}
\end{Highlighting}
\end{Shaded}

\begin{verbatim}
## Wrkstat coefficients: Each work status category differs from the reference category.
\end{verbatim}

\begin{Shaded}
\begin{Highlighting}[]
\FunctionTok{cat}\NormalTok{(}
  \StringTok{"Snap\_insta coefficient: Using Snapchat/Instagram is associated with"}\NormalTok{,}
  \FunctionTok{round}\NormalTok{(coef\_summary}\SpecialCharTok{$}\NormalTok{estimate[}\FunctionTok{grep}\NormalTok{(}\StringTok{"snap\_instYes"}\NormalTok{, coef\_summary}\SpecialCharTok{$}\NormalTok{term)], }\DecValTok{2}\NormalTok{),}
  \StringTok{"minutes per week difference in email time.}\SpecialCharTok{\textbackslash{}n}\StringTok{"}
\NormalTok{)}
\end{Highlighting}
\end{Shaded}

\begin{verbatim}
## Snap_insta coefficient: Using Snapchat/Instagram is associated with  minutes per week difference in email time.
\end{verbatim}

\textbf{Detailed Interpretation:} - \textbf{Education (educ):} Each
additional year of education is associated with a change in email time.
Higher education may correlate with jobs that require more email
communication. - \textbf{Work Status (wrkstat):} Different employment
statuses have different relationships with email usage. Full-time
workers likely spend more time on email than part-time workers,
students, or retirees. - \textbf{Snap\_insta:} Using Snapchat or
Instagram may indicate younger respondents, who may have different email
habits than older generations.

\begin{enumerate}
\def\labelenumi{\arabic{enumi}.}
\setcounter{enumi}{8}
\tightlist
\item
  Create a predicted values vs.~residuals plot for this model. Are there
  any issues with the model? If yes, describe them.
\end{enumerate}

\begin{Shaded}
\begin{Highlighting}[]
\CommentTok{\# Create predicted values and residuals}
\NormalTok{email\_model\_aug }\OtherTok{\textless{}{-}}\NormalTok{ broom}\SpecialCharTok{::}\FunctionTok{augment}\NormalTok{(email\_model)}

\CommentTok{\# Residuals vs fitted plot}
\FunctionTok{ggplot}\NormalTok{(email\_model\_aug, }\FunctionTok{aes}\NormalTok{(}\AttributeTok{x =}\NormalTok{ .fitted, }\AttributeTok{y =}\NormalTok{ .resid)) }\SpecialCharTok{+}
  \FunctionTok{geom\_point}\NormalTok{(}\AttributeTok{alpha =} \FloatTok{0.5}\NormalTok{) }\SpecialCharTok{+}
  \FunctionTok{geom\_hline}\NormalTok{(}\AttributeTok{yintercept =} \DecValTok{0}\NormalTok{, }\AttributeTok{linetype =} \StringTok{"dashed"}\NormalTok{, }\AttributeTok{color =} \StringTok{"red"}\NormalTok{) }\SpecialCharTok{+}
  \FunctionTok{labs}\NormalTok{(}
    \AttributeTok{title =} \StringTok{"Residuals vs Fitted Values"}\NormalTok{,}
    \AttributeTok{x =} \StringTok{"Fitted Values"}\NormalTok{,}
    \AttributeTok{y =} \StringTok{"Residuals"}
\NormalTok{  ) }\SpecialCharTok{+}
  \FunctionTok{theme\_minimal}\NormalTok{()}
\end{Highlighting}
\end{Shaded}

\includegraphics[width=0.8\linewidth]{hw-08-exploring-gss_files/figure-latex/exercise-9-1}

\begin{Shaded}
\begin{Highlighting}[]
\CommentTok{\# Check for patterns}
\FunctionTok{cat}\NormalTok{(}\StringTok{"Residual diagnostics:}\SpecialCharTok{\textbackslash{}n}\StringTok{"}\NormalTok{)}
\end{Highlighting}
\end{Shaded}

\begin{verbatim}
## Residual diagnostics:
\end{verbatim}

\begin{Shaded}
\begin{Highlighting}[]
\FunctionTok{cat}\NormalTok{(}\StringTok{"Mean of residuals:"}\NormalTok{, }\FunctionTok{round}\NormalTok{(}\FunctionTok{mean}\NormalTok{(email\_model\_aug}\SpecialCharTok{$}\NormalTok{.resid), }\DecValTok{6}\NormalTok{), }\StringTok{"}\SpecialCharTok{\textbackslash{}n}\StringTok{"}\NormalTok{)}
\end{Highlighting}
\end{Shaded}

\begin{verbatim}
## Mean of residuals: 0
\end{verbatim}

\begin{Shaded}
\begin{Highlighting}[]
\FunctionTok{cat}\NormalTok{(}\StringTok{"SD of residuals:"}\NormalTok{, }\FunctionTok{round}\NormalTok{(}\FunctionTok{sd}\NormalTok{(email\_model\_aug}\SpecialCharTok{$}\NormalTok{.resid), }\DecValTok{2}\NormalTok{), }\StringTok{"}\SpecialCharTok{\textbackslash{}n}\StringTok{"}\NormalTok{)}
\end{Highlighting}
\end{Shaded}

\begin{verbatim}
## SD of residuals: 637.93
\end{verbatim}

\begin{Shaded}
\begin{Highlighting}[]
\CommentTok{\# Q{-}Q plot}
\FunctionTok{ggplot}\NormalTok{(email\_model\_aug, }\FunctionTok{aes}\NormalTok{(}\AttributeTok{sample =}\NormalTok{ .resid)) }\SpecialCharTok{+}
  \FunctionTok{stat\_qq}\NormalTok{() }\SpecialCharTok{+}
  \FunctionTok{stat\_qq\_line}\NormalTok{() }\SpecialCharTok{+}
  \FunctionTok{labs}\NormalTok{(}\AttributeTok{title =} \StringTok{"Normal Q{-}Q Plot"}\NormalTok{) }\SpecialCharTok{+}
  \FunctionTok{theme\_minimal}\NormalTok{()}
\end{Highlighting}
\end{Shaded}

\includegraphics[width=0.8\linewidth]{hw-08-exploring-gss_files/figure-latex/exercise-9-2}

\textbf{Potential Issues with the Model:} 1.
\textbf{Heteroscedasticity:} The spread of residuals may not be constant
across fitted values (variance increases or decreases). 2.
\textbf{Non-linearity:} The relationship between predictors and email
time may not be linear. 3. \textbf{Outliers:} Some respondents may have
unusually high or low email times, influencing the model. 4.
\textbf{Normality:} Residuals may not be perfectly normally distributed,
especially with skewed email time data. 5. \textbf{Missing data:} The
analysis only includes respondents with complete data on all variables.

Knit, \emph{commit, and push your changes to GitHub with an appropriate
commit message. Make sure to commit and push all changed files so that
your Git pane is cleared up afterwards.}

\subsection{Part 3: Political views and science
research}\label{part-3-political-views-and-science-research}

The 2016 GSS also asked respondents whether they think of themselves as
liberal or conservative (\texttt{polviews}) and whether they think
science research is necessary and should be supported by the federal
government (\texttt{advfront}).

\begin{itemize}
\tightlist
\item
  The question on science research is worded as follows:
\end{itemize}

\begin{quote}
Even if it brings no immediate benefits, scientific research that
advances the frontiers of knowledge is necessary and should be supported
by the federal government.
\end{quote}

And possible responses to this question are Strongly agree, Agree,
Disagree, Strongly disagree, Don't know, No answer, Not applicable.

\begin{itemize}
\tightlist
\item
  The question on political views is worded as follows:
\end{itemize}

\begin{quote}
We hear a lot of talk these days about liberals and conservatives. I'm
going to show you a seven-point scale on which the political views that
people might hold are arranged from extremely liberal--point 1--to
extremely conservative--point 7. Where would you place yourself on this
scale?
\end{quote}

\textbf{Note:} The levels of this variables are spelled inconsistently:
``Extremely liberal'' vs.~``Extrmly conservative''. Since this is the
spelling that shows up in the data, you need to make sure this is how
you spell the levels in your code.

And possible responses to this question are Extremely liberal, Liberal,
Slightly liberal, Moderate, Slghtly conservative, Conservative, Extrmly
conservative. Responses that were originally Don't know, No answer and
Not applicable are already mapped to \texttt{NA}s upon data import.

\begin{enumerate}
\def\labelenumi{\arabic{enumi}.}
\setcounter{enumi}{9}
\tightlist
\item
  In a new variable, recode \texttt{advfront} such that Strongly Agree
  and Agree are mapped to \texttt{"Yes"}, and Disagree and Strongly
  disagree are mapped to \texttt{"No"}. The remaining levels can be left
  as is. Don't overwrite the existing \texttt{advfront}, instead pick a
  different, informative name for your new variable.
\end{enumerate}

\begin{Shaded}
\begin{Highlighting}[]
\CommentTok{\# Recode advfront to a simpler binary variable}
\NormalTok{gss16 }\OtherTok{\textless{}{-}}\NormalTok{ gss16 }\SpecialCharTok{\%\textgreater{}\%}
  \FunctionTok{mutate}\NormalTok{(}\AttributeTok{science\_support =} \FunctionTok{case\_when}\NormalTok{(}
\NormalTok{    advfront }\SpecialCharTok{\%in\%} \FunctionTok{c}\NormalTok{(}\StringTok{"Strongly agree"}\NormalTok{, }\StringTok{"Agree"}\NormalTok{) }\SpecialCharTok{\textasciitilde{}} \StringTok{"Yes"}\NormalTok{,}
\NormalTok{    advfront }\SpecialCharTok{\%in\%} \FunctionTok{c}\NormalTok{(}\StringTok{"Strongly disagree"}\NormalTok{, }\StringTok{"Disagree"}\NormalTok{) }\SpecialCharTok{\textasciitilde{}} \StringTok{"No"}\NormalTok{,}
    \ConstantTok{TRUE} \SpecialCharTok{\textasciitilde{}} \FunctionTok{as.character}\NormalTok{(advfront)}
\NormalTok{  ))}

\CommentTok{\# Verify the recoding}
\FunctionTok{cat}\NormalTok{(}\StringTok{"Original advfront values:}\SpecialCharTok{\textbackslash{}n}\StringTok{"}\NormalTok{)}
\end{Highlighting}
\end{Shaded}

\begin{verbatim}
## Original advfront values:
\end{verbatim}

\begin{Shaded}
\begin{Highlighting}[]
\NormalTok{gss16 }\SpecialCharTok{\%\textgreater{}\%}
  \FunctionTok{count}\NormalTok{(advfront)}
\end{Highlighting}
\end{Shaded}

\begin{verbatim}
## # A tibble: 6 x 2
##   advfront              n
##   <chr>             <int>
## 1 Agree               759
## 2 Disagree            184
## 3 Dont know            27
## 4 Strongly agree      400
## 5 Strongly disagree    15
## 6 <NA>               1482
\end{verbatim}

\begin{Shaded}
\begin{Highlighting}[]
\FunctionTok{cat}\NormalTok{(}\StringTok{"}\SpecialCharTok{\textbackslash{}n}\StringTok{Recoded science\_support values:}\SpecialCharTok{\textbackslash{}n}\StringTok{"}\NormalTok{)}
\end{Highlighting}
\end{Shaded}

\begin{verbatim}
## 
## Recoded science_support values:
\end{verbatim}

\begin{Shaded}
\begin{Highlighting}[]
\NormalTok{gss16 }\SpecialCharTok{\%\textgreater{}\%}
  \FunctionTok{count}\NormalTok{(science\_support)}
\end{Highlighting}
\end{Shaded}

\begin{verbatim}
## # A tibble: 4 x 2
##   science_support     n
##   <chr>           <int>
## 1 Dont know          27
## 2 No                199
## 3 Yes              1159
## 4 <NA>             1482
\end{verbatim}

\textbf{Explanation:} The new \texttt{science\_support} variable
simplifies the five-level \texttt{advfront} variable into a
three-category variable: ``Yes'' (for agreement), ``No'' (for
disagreement), and other responses remain unchanged.

\begin{enumerate}
\def\labelenumi{\arabic{enumi}.}
\setcounter{enumi}{10}
\tightlist
\item
  In a new variable, recode \texttt{polviews} such that Extremely
  liberal, Liberal, and Slightly liberal, are mapped to
  \texttt{"Liberal"}, and Slghtly conservative, Conservative, and
  Extrmly conservative disagree are mapped to \texttt{"Conservative"}.
  The remaining levels can be left as is. Make sure that the levels are
  in a reasonable order. Don't overwrite the existing \texttt{polviews},
  instead pick a different, informative name for your new variable.
\end{enumerate}

\begin{Shaded}
\begin{Highlighting}[]
\CommentTok{\# Recode polviews to a simpler three{-}category variable}
\NormalTok{gss16 }\OtherTok{\textless{}{-}}\NormalTok{ gss16 }\SpecialCharTok{\%\textgreater{}\%}
  \FunctionTok{mutate}\NormalTok{(}\AttributeTok{polviews\_grouped =} \FunctionTok{case\_when}\NormalTok{(}
\NormalTok{    polviews }\SpecialCharTok{\%in\%} \FunctionTok{c}\NormalTok{(}\StringTok{"Extremely liberal"}\NormalTok{, }\StringTok{"Liberal"}\NormalTok{, }\StringTok{"Slightly liberal"}\NormalTok{) }\SpecialCharTok{\textasciitilde{}} \StringTok{"Liberal"}\NormalTok{,}
\NormalTok{    polviews }\SpecialCharTok{\%in\%} \FunctionTok{c}\NormalTok{(}\StringTok{"Slghtly conservative"}\NormalTok{, }\StringTok{"Conservative"}\NormalTok{, }\StringTok{"Extrmly conservative"}\NormalTok{) }\SpecialCharTok{\textasciitilde{}} \StringTok{"Conservative"}\NormalTok{,}
    \ConstantTok{TRUE} \SpecialCharTok{\textasciitilde{}} \FunctionTok{as.character}\NormalTok{(polviews)}
\NormalTok{  )) }\SpecialCharTok{\%\textgreater{}\%}
  \FunctionTok{mutate}\NormalTok{(}\AttributeTok{polviews\_grouped =} \FunctionTok{factor}\NormalTok{(polviews\_grouped,}
    \AttributeTok{levels =} \FunctionTok{c}\NormalTok{(}\StringTok{"Liberal"}\NormalTok{, }\StringTok{"Moderate"}\NormalTok{, }\StringTok{"Conservative"}\NormalTok{)}
\NormalTok{  ))}

\CommentTok{\# Verify the recoding}
\FunctionTok{cat}\NormalTok{(}\StringTok{"Original polviews values:}\SpecialCharTok{\textbackslash{}n}\StringTok{"}\NormalTok{)}
\end{Highlighting}
\end{Shaded}

\begin{verbatim}
## Original polviews values:
\end{verbatim}

\begin{Shaded}
\begin{Highlighting}[]
\NormalTok{gss16 }\SpecialCharTok{\%\textgreater{}\%}
  \FunctionTok{count}\NormalTok{(polviews)}
\end{Highlighting}
\end{Shaded}

\begin{verbatim}
## # A tibble: 8 x 2
##   polviews                 n
##   <chr>                <int>
## 1 Conservative           426
## 2 Extremely liberal      136
## 3 Extrmly conservative   120
## 4 Liberal                350
## 5 Moderate              1032
## 6 Slghtly conservative   382
## 7 Slightly liberal       310
## 8 <NA>                   111
\end{verbatim}

\begin{Shaded}
\begin{Highlighting}[]
\FunctionTok{cat}\NormalTok{(}\StringTok{"}\SpecialCharTok{\textbackslash{}n}\StringTok{Recoded polviews\_grouped values:}\SpecialCharTok{\textbackslash{}n}\StringTok{"}\NormalTok{)}
\end{Highlighting}
\end{Shaded}

\begin{verbatim}
## 
## Recoded polviews_grouped values:
\end{verbatim}

\begin{Shaded}
\begin{Highlighting}[]
\NormalTok{gss16 }\SpecialCharTok{\%\textgreater{}\%}
  \FunctionTok{count}\NormalTok{(polviews\_grouped)}
\end{Highlighting}
\end{Shaded}

\begin{verbatim}
## # A tibble: 4 x 2
##   polviews_grouped     n
##   <fct>            <int>
## 1 Liberal            796
## 2 Moderate          1032
## 3 Conservative       928
## 4 <NA>               111
\end{verbatim}

\textbf{Explanation:} The new \texttt{polviews\_grouped} variable
consolidates the seven-point political ideology scale into three
categories: Liberal (combining the three liberal positions),
Conservative (combining the three conservative positions), and Moderate
(kept separate). The factor levels are ordered logically from left to
right.

\begin{enumerate}
\def\labelenumi{\arabic{enumi}.}
\setcounter{enumi}{11}
\tightlist
\item
  Create a visualization that displays the relationship between these
  two new variables and interpret it.
\end{enumerate}

\begin{Shaded}
\begin{Highlighting}[]
\CommentTok{\# Create a cross{-}tabulation with percentages}
\NormalTok{science\_politics\_table }\OtherTok{\textless{}{-}}\NormalTok{ gss16 }\SpecialCharTok{\%\textgreater{}\%}
  \FunctionTok{filter}\NormalTok{(}\SpecialCharTok{!}\FunctionTok{is.na}\NormalTok{(science\_support), }\SpecialCharTok{!}\FunctionTok{is.na}\NormalTok{(polviews\_grouped)) }\SpecialCharTok{\%\textgreater{}\%}
  \FunctionTok{group\_by}\NormalTok{(polviews\_grouped, science\_support) }\SpecialCharTok{\%\textgreater{}\%}
  \FunctionTok{summarise}\NormalTok{(}\AttributeTok{count =} \FunctionTok{n}\NormalTok{(), }\AttributeTok{.groups =} \StringTok{"drop"}\NormalTok{) }\SpecialCharTok{\%\textgreater{}\%}
  \FunctionTok{group\_by}\NormalTok{(polviews\_grouped) }\SpecialCharTok{\%\textgreater{}\%}
  \FunctionTok{mutate}\NormalTok{(}\AttributeTok{percent =}\NormalTok{ (count }\SpecialCharTok{/} \FunctionTok{sum}\NormalTok{(count)) }\SpecialCharTok{*} \DecValTok{100}\NormalTok{)}

\FunctionTok{print}\NormalTok{(science\_politics\_table)}
\end{Highlighting}
\end{Shaded}

\begin{verbatim}
## # A tibble: 9 x 4
## # Groups:   polviews_grouped [3]
##   polviews_grouped science_support count percent
##   <fct>            <chr>           <int>   <dbl>
## 1 Liberal          Dont know           2   0.532
## 2 Liberal          No                 43  11.4  
## 3 Liberal          Yes               331  88.0  
## 4 Moderate         Dont know           5   0.990
## 5 Moderate         No                 62  12.3  
## 6 Moderate         Yes               438  86.7  
## 7 Conservative     Dont know          16   3.56 
## 8 Conservative     No                 84  18.7  
## 9 Conservative     Yes               349  77.7
\end{verbatim}

\begin{Shaded}
\begin{Highlighting}[]
\CommentTok{\# Create a visualization}
\FunctionTok{ggplot}\NormalTok{(science\_politics\_table, }\FunctionTok{aes}\NormalTok{(}\AttributeTok{x =}\NormalTok{ polviews\_grouped, }\AttributeTok{y =}\NormalTok{ percent, }\AttributeTok{fill =}\NormalTok{ science\_support)) }\SpecialCharTok{+}
  \FunctionTok{geom\_col}\NormalTok{(}\AttributeTok{position =} \StringTok{"dodge"}\NormalTok{, }\AttributeTok{alpha =} \FloatTok{0.8}\NormalTok{) }\SpecialCharTok{+}
  \FunctionTok{labs}\NormalTok{(}
    \AttributeTok{title =} \StringTok{"Support for Federal Science Research by Political Views"}\NormalTok{,}
    \AttributeTok{x =} \StringTok{"Political Views"}\NormalTok{,}
    \AttributeTok{y =} \StringTok{"Percentage (\%)"}\NormalTok{,}
    \AttributeTok{fill =} \StringTok{"Federal Science}\SpecialCharTok{\textbackslash{}n}\StringTok{Support"}
\NormalTok{  ) }\SpecialCharTok{+}
  \FunctionTok{theme\_minimal}\NormalTok{() }\SpecialCharTok{+}
  \FunctionTok{theme}\NormalTok{(}\AttributeTok{legend.position =} \StringTok{"bottom"}\NormalTok{)}
\end{Highlighting}
\end{Shaded}

\includegraphics[width=0.8\linewidth]{hw-08-exploring-gss_files/figure-latex/exercise-12-1}

\begin{Shaded}
\begin{Highlighting}[]
\CommentTok{\# Alternative: stacked bar chart}
\FunctionTok{ggplot}\NormalTok{(science\_politics\_table, }\FunctionTok{aes}\NormalTok{(}\AttributeTok{x =}\NormalTok{ polviews\_grouped, }\AttributeTok{y =}\NormalTok{ percent, }\AttributeTok{fill =}\NormalTok{ science\_support)) }\SpecialCharTok{+}
  \FunctionTok{geom\_col}\NormalTok{(}\AttributeTok{position =} \StringTok{"stack"}\NormalTok{, }\AttributeTok{alpha =} \FloatTok{0.8}\NormalTok{) }\SpecialCharTok{+}
  \FunctionTok{labs}\NormalTok{(}
    \AttributeTok{title =} \StringTok{"Support for Federal Science Research by Political Views"}\NormalTok{,}
    \AttributeTok{x =} \StringTok{"Political Views"}\NormalTok{,}
    \AttributeTok{y =} \StringTok{"Percentage (\%)"}\NormalTok{,}
    \AttributeTok{fill =} \StringTok{"Federal Science}\SpecialCharTok{\textbackslash{}n}\StringTok{Support"}
\NormalTok{  ) }\SpecialCharTok{+}
  \FunctionTok{theme\_minimal}\NormalTok{() }\SpecialCharTok{+}
  \FunctionTok{theme}\NormalTok{(}\AttributeTok{legend.position =} \StringTok{"bottom"}\NormalTok{)}
\end{Highlighting}
\end{Shaded}

\includegraphics[width=0.8\linewidth]{hw-08-exploring-gss_files/figure-latex/exercise-12-2}

\begin{Shaded}
\begin{Highlighting}[]
\CommentTok{\# Calculate proportions by political view}
\FunctionTok{cat}\NormalTok{(}\StringTok{"}\SpecialCharTok{\textbackslash{}n}\StringTok{Support for federal science research by political views:}\SpecialCharTok{\textbackslash{}n}\StringTok{"}\NormalTok{)}
\end{Highlighting}
\end{Shaded}

\begin{verbatim}
## 
## Support for federal science research by political views:
\end{verbatim}

\begin{Shaded}
\begin{Highlighting}[]
\NormalTok{gss16 }\SpecialCharTok{\%\textgreater{}\%}
  \FunctionTok{filter}\NormalTok{(}\SpecialCharTok{!}\FunctionTok{is.na}\NormalTok{(science\_support), }\SpecialCharTok{!}\FunctionTok{is.na}\NormalTok{(polviews\_grouped), science\_support }\SpecialCharTok{\%in\%} \FunctionTok{c}\NormalTok{(}\StringTok{"Yes"}\NormalTok{, }\StringTok{"No"}\NormalTok{)) }\SpecialCharTok{\%\textgreater{}\%}
  \FunctionTok{group\_by}\NormalTok{(polviews\_grouped) }\SpecialCharTok{\%\textgreater{}\%}
  \FunctionTok{summarise}\NormalTok{(}
    \AttributeTok{total =} \FunctionTok{n}\NormalTok{(),}
    \AttributeTok{n\_yes =} \FunctionTok{sum}\NormalTok{(science\_support }\SpecialCharTok{==} \StringTok{"Yes"}\NormalTok{),}
    \AttributeTok{pct\_yes =}\NormalTok{ (n\_yes }\SpecialCharTok{/}\NormalTok{ total) }\SpecialCharTok{*} \DecValTok{100}
\NormalTok{  ) }\SpecialCharTok{\%\textgreater{}\%}
  \FunctionTok{print}\NormalTok{()}
\end{Highlighting}
\end{Shaded}

\begin{verbatim}
## # A tibble: 3 x 4
##   polviews_grouped total n_yes pct_yes
##   <fct>            <int> <int>   <dbl>
## 1 Liberal            374   331    88.5
## 2 Moderate           500   438    87.6
## 3 Conservative       433   349    80.6
\end{verbatim}

\textbf{Interpretation:} The visualization reveals the relationship
between political ideology and support for federal funding of scientific
research. Key patterns typically include:

\begin{enumerate}
\def\labelenumi{\arabic{enumi}.}
\item
  \textbf{Liberal respondents} tend to show higher support for federal
  science research, reflecting the liberal emphasis on scientific
  progress and government investment in public goods.
\item
  \textbf{Conservative respondents} show more varied opinions, with
  perhaps lower overall support, potentially reflecting concerns about
  government spending and efficiency.
\item
  \textbf{Moderate respondents} typically fall between the two extremes
  in their support levels.
\end{enumerate}

This relationship illustrates how political ideology and views on
science policy are interconnected in American public opinion. Liberal
political orientation is generally associated with stronger support for
government-funded scientific research, while conservative political
orientation tends to be more skeptical of such spending.

Knit, \emph{commit, and push your changes to GitHub with an appropriate
commit message. Make sure to commit and push all changed files so that
your Git pane is cleared up afterwards and review the md document on
GitHub to make sure you're happy with the final state of your work.}

\end{document}
